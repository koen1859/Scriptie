\documentclass[a4paper,11pt]{article}

\usepackage[english]{babel}
\usepackage{mathtools,amsthm,amssymb,amsmath}
\usepackage{booktabs} % For better-looking tables
\usepackage{dcolumn}  % For aligning decimal points in tables
\usepackage{bm}
\usepackage{hyperref}
\usepackage{ marvosym }
\usepackage{eurosym}
\usepackage{bm}
\usepackage{graphicx}
\usepackage{subfig}
\usepackage{tabularx}
\usepackage{longtable}
\usepackage[longnamesfirst]{natbib}
\newcommand{\N}[1]{\mathcal{N}\left(#1\right)}
\newcommand{\Z}{\mathbb{Z}}\DeclareMathOperator*{\argmin}{arg\,min}
\newcommand{\abs}[1]{\left\vert#1\right\vert}
\newcommand{\given}{\,\middle|\,}
\newcommand{\Bern}[1]{\mathrm{Bern}(#1)}
\newcommand{\Bin}[1]{\mathrm{Bin}(#1)}
\newcommand{\Exp}[1]{\mathrm{Exp}(#1)}
\newcommand{\FS}[1]{\mathrm{FS}(#1)}
\newcommand{\Geo}[1]{\mathrm{Geo}(#1)}
\newcommand{\Norm}[1]{\mathrm{Norm}(#1)}
\newcommand{\Pois}[1]{\mathrm{Pois}(#1)}
\newcommand{\Unif}[1]{\mathrm{Unif}(#1)}
\newcommand{\E}[1]{\,\mathsf{E}\left[#1\right]}
\newcommand{\EE}[2]{\,\mathsf{E}_{#1}\left[#2\right]}
\newcommand{\V}[1]{\,\mathsf{V}\left[#1\right]}
\newcommand{\cov}[1]{\,\mathsf{Cov}\left[#1\right]}
\renewcommand{\d}[1]{\,\textrm{d}#1}
\newcommand{\1}[1]{\,I_{#1}} % indicator
\renewcommand{\P}[1]{\,\mathbb{P}\left\{#1\right\}}
\newcommand{\fat}[1]{\ThisStyle{\hstretch{1.2}{\ooalign{%
        \kern.46pt$\SavedStyle#1$\cr\kern.33pt$\SavedStyle#1$\cr%
\kern.2pt$\SavedStyle#1$\cr$\SavedStyle#1$}}}}
\renewcommand{\ln}[1]{\,\mathrm{ln}\left[#1\right]}
\newcommand*{\B}[1]{\ifmmode\bm{#1}\else\textbf{#1}\fi}
\title{Research Proposal\\
  Assessment of approximation method for TSP path length on a road
network: a simulation study}
\author{Koen Stevens\\
S5302137}
\date{\today}
\begin{document}
\maketitle
\section{Introduction}
The Traveling Salesman Problem is an important problem in operations
research. It is particularly relevant for last-mile carriers and
other logistics companies where efficient routing directly impacts
cost, time and service quality. Efficient approximation methods are
important for practical applications where exact solutions are too
computationally intensive to conduct.
\section{Literature Review}
\cite{beardwood1959shortest} proposed the formula
\begin{align}
	L=\beta\sqrt{nA}
	\label{Beardwood Thm}
\end{align}
as a predictor for the length of the shortest TSP path measured by
euclidean distance through $n$ random locations inside an area in
$\mathbb{R}^2$ with area $A$, where $\beta$ is some constant. This
constant has been estimated in various research, with estimations
ranging between $0.7$ and $1$. \cite{doi:10.1177/03611981211049433}
provides a list of this research.
\section{Problem Formulation}
The main objective of this research is to estimate $\beta$ and assess
how well formula \ref{Beardwood Thm}, approximates the length of the
optimal TSP path in a real-world area.
\section{Research Question}
A few questions arise in order to tackle this problem:
\begin{itemize}
	\item What is $\beta$ in formula \ref{Beardwood Thm} when applied
	      to a real road network?
	\item How well does this formula predict the length of the TSP path
	      on a road network?
	\item Do the value of $\beta$ and the performance of the formula
	      vary across different types of areas (rural or urban, grid-like or organic)?
\end{itemize}
\section{Hypotheses}
I expect to find a higher $\beta$ than the $\beta$ that has been
estimated previously in simulations where the $\mathbb{R}^2$ was used
and no road network was considered, since traveling over a road
network is always just as long or longer than traveling in a straight
line. I also expect this $\beta$ to vary when different types of
areas are considered.
\section{Data}
The data used is
\href{https://www.openstreetmap.org/#map=12/53.2184/6.5702}{OpenStreetMap}
data for the
\href{https://download.geofabrik.de/europe/netherlands/groningen.html}{Province
	of Groningen}.
\section{Methodology}
By comparing the estimated and actual TSP path lengths for simulated
sets of locations within the area, this study aims to evaluate the
accuracy of this formula in a real-world setting.
\begin{enumerate}
	\item Make a connected graph of the road network of Groningen,
	      using the igraph module in python.
	\item Simulate locations on this graph.
	\item Make a distance matrix of these locations.
	\item Use an efficient method to find the shortest TSP path (LKH).
	\item Compute the length of this TSP path.
	\item Conduct 2-5 a number of times and estimate $\beta$.
	\item Assess how good of a predictor the formula $L=\beta\sqrt{nA}$ is.
\end{enumerate}
\bibliography{literature}
\bibliographystyle{rug-econometrics}
\section{Appendix}
\href{https://github.com/koen1859/Scriptie}{Some code of the current
	progress.}\\
\href{https://koenstevens.nl/?page_id=182}{Some visualizations of the
	road network and paths generated with shortest path algorithm.}
\end{document}
