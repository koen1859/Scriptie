\documentclass[12pt]{article}

\usepackage[a4paper,margin=1in]{geometry}
\usepackage{setspace}
\usepackage{lmodern}
\usepackage[english]{babel}
\usepackage{mathtools,amsthm,amssymb,amsmath}
\numberwithin{equation}{section}
\usepackage{booktabs} % For better-looking tables
\usepackage{dcolumn}  % For aligning decimal points in tables
\usepackage{bm}
\usepackage{hyperref}
\usepackage{ marvosym }
\usepackage{eurosym}
\usepackage{bm}
\usepackage{graphicx}
\usepackage[longnamesfirst]{natbib}
\usepackage{float}
\usepackage{longtable}
\usepackage{caption}

\setlength{\parindent}{0pt}     % No paragraph indentation
\setlength{\parskip}{1em}       % Add vertical space between paragraphs

\newcommand{\N}[1]{\mathcal{N}\left(#1\right)}
\newcommand{\Z}{\mathbb{Z}}
\DeclareMathOperator*{\argmin}{arg\,min}
\DeclareMathOperator*{\argmax}{arg\,max}
\newcommand{\abs}[1]{\left\vert#1\right\vert}
\newcommand{\given}{\,\middle|\,}
\newcommand{\Bern}[1]{\mathrm{Bern}(#1)}
\newcommand{\Bin}[1]{\mathrm{Bin}(#1)}
\newcommand{\Exp}[1]{\mathrm{Exp}(#1)}
\newcommand{\FS}[1]{\mathrm{FS}(#1)}
\newcommand{\Geo}[1]{\mathrm{Geo}(#1)}
\newcommand{\Norm}[1]{\mathrm{Norm}(#1)}
\newcommand{\Pois}[1]{\mathrm{Pois}(#1)}
\newcommand{\Unif}[1]{\mathrm{Unif}(#1)}
\newcommand{\E}[1]{\,\mathsf{E}\left[#1\right]}
\newcommand{\EE}[2]{\,\mathsf{E}_{#1}\left[#2\right]}
\newcommand{\V}[1]{\,\mathsf{V}\left[#1\right]}
\newcommand{\cov}[1]{\,\mathsf{Cov}\left[#1\right]}
\renewcommand{\d}[1]{\,\textrm{d}#1}
\newcommand{\1}[1]{\,I_{#1}} % indicator
\renewcommand{\P}[1]{\,\mathbb{P}\left\{#1\right\}}
\newcommand{\fat}[1]{\ThisStyle{\hstretch{1.2}{\ooalign{%
  \kern.46pt$\SavedStyle#1$\cr\kern.33pt$\SavedStyle#1$\cr%
  \kern.2pt$\SavedStyle#1$\cr$\SavedStyle#1$}}}}
\renewcommand{\ln}[1]{\,\mathrm{ln}\left[#1\right]}
\newcommand*{\B}[1]{\ifmmode\bm{#1}\else\textbf{#1}\fi}
\pagestyle{empty}

\title{Assessment of an approximation method for TSP path length on road networks}
\author{Koen Stevens}
\date{\today}
\begin{document}
\begin{titlepage}
	\centering
	\vspace*{0.3\textheight}
	{\LARGE\bfseries Assessment of an approximation method for TSP path length on road networks \par}
	\vspace{2cm}
	{\Large Koen Stevens \par}
	\vfill
	{\large \today \par} % or hardcode the date
\end{titlepage}
\clearpage
\thispagestyle{empty}
\vspace*{0.3\textheight}
{\Large Bachelor's Thesis Econometrics \par}
\vspace{1cm}
{\large Supervisor: dr. N.D. van Foreest \par}
\vspace{0.5cm}
{\large Second assessor:  \par}
\clearpage
\pagenumbering{arabic}
\pagestyle{plain}
\begin{center}
	\LARGE \textbf{Assessment of an approximation method for TSP path length on road networks} \\[1.5ex]
	\large \textit{Koen Stevens}
\end{center}
\begin{abstract}

\end{abstract}
\section{Introduction}
The Traveling Salesman Problem (TSP) is an important problem in operations research.
It is particularly relevant for last-mile carriers and other logistics companies where efficient
routing directly impacts cost, time and service quality. Since the number of parcels worldwide has
increased between 2013 and 2022 and is expected to keep increasing \cite{statista}, the need for
fast, scalable route planning methods becomes ever more pressing.

The TSP is an NP-hard problem, it is computationally intensive to find the exact solution for
large instances. In many real-world scenarios, the exact optimal routes may not be needed, but
instead a rough, reliable estimate of the optimal route length. For instance, consider a postal delivery company.
This firm may need to assign a certain amount of deliveries or a certain area to each postman.
Reliable estimates for the route length can provide valuable information for making such decisions.

Efficient approximation methods provide a solution for such practical applications where exact
solutions are too computationally intensive to conduct or not feasible due to insufficient data.
These methods aim at approximating the expected optimal total travel time or distance, while using
minimal data and computational effort.

There is extensive research on such approximation methods and how they perform in the Euclidean
plane.
Consider $n$ uniformly drawn locations from some area in $\mathbb{R}^2$ with area $A$.
\cite{beardwood1959shortest} prove the relation:
\begin{align}
	L \to \beta \sqrt{nA}, \quad \text{as } n \to \infty
	\label{eq:beardwood}
\end{align}
as an estimation for the length of the shortest TSP path measured by Euclidean distance through
these random locations, where $\beta$ is some proportionality constant. This formula is a very
elegant result, and it requires very little data. However, its assumptions,
uniform random locations and euclidean space differ from real-world applications, which are defined
by complex geographic features, such as road networks.

This research investigates how well this approximation method performs when considering real road
networks. Using OpenStreetMap data, TSP instances are simulated in a wide variety of different urban areas
in the Netherlands, then solve these for the actual shortest paths using the Lin–Kernighan \cite{lin1973effective} heuristic.
Then the $\beta$ from equation \ref{eq:beardwood} is estimated and the performance of this
formula is analyzed. Additionally, the results for $\beta$ and the performance across the selected
areas is compared.

In section 2 a deep dive in the context and previous research in this field is provided.
In section 3 the experimental design is documented.
\section{Literature Review}
In this section the existing literature on the Beardwood formula and some applications,
and on the Lin-Kernighan heuristic and its implementations is reviewed.
\subsection{Applications of the Beardwood formula}
This research concerns the performance of formula \ref{eq:beardwood} for reasonable amounts of
locations a delivery person can visit in a workday, say $10\leq n\leq90$. \cite{lei2015dynamic}
estimates the values of $\beta$ for a selection of values for $n$.
In their research, the points were generated uniformly and the $L_2$ distance metric was used.
Table \ref{tab:beta-values} lists the results.
\begin{table}[H]
	\centering
	\caption{Empirical estimates of $\beta$ as a function of $n$, $20 \leq n \leq 90$\\
		(\cite{lei2015dynamic})}
	\label{tab:beta-values}
	\begin{tabular}{cc}
		\toprule
		$n$ & $\beta(n)$ \\
		\midrule
		20  & 0.8584265  \\
		30  & 0.8269698  \\
		40  & 0.8129900  \\
		50  & 0.7994125  \\
		60  & 0.7908632  \\
		70  & 0.7817751  \\
		80  & 0.7775367  \\
		90  & 0.7773827  \\
		\bottomrule
	\end{tabular}
\end{table}
\cite{figliozzi2008planning} is the first research to apply approximation formulas to real-world
instances of TSPs (and VRPs (Vehicle Routing Problems)). An extension of formula
\ref{eq:beardwood} that works for VRPs is assessed in a real-world setting. It is found that this
model has an $R^2$ of 0.99 and MAPE (Mean Absolute Prediction Error) of 4.2\%. This prediction error
is slightly higher than when it is applied to a setting where Euclidean distances are considered (3.0\%),
but the formula still performs well (\cite{figliozzi2008planning}).

\cite{merchan2019empirical} use circuity factors to measure the relative detour incurred for
traveling in a road network, compared to the Euclidean distance. This circuity factor is defined
as, where $p$ and $q$ are locations:
\begin{align}
	c = \frac{d_{c}(p,q)}{d_{L_{2}}(p,q)}
	\label{eq:circuity}
\end{align}
By construction, $c$ is greater or equal to 1, a value closer to 1 indicates a more efficient network. Then, $\beta_c$
is estimated by $\beta_c=c\beta$. This value $c$, is estimated for three different areas in
São Paulo, for which the results are listed in table \ref{tab:beta-merchan}. These values indicate
real travel distances are on average 2.76 times longer in area 1 compared to the $L_2$ metric.
These values were obtained by uniformly generating $n$ locations (for $n$ ranging from 3 to 250),
computing near-optimal tour lengths under the Euclidean metric, and solving for $\beta$, then
scaling by the empirical circuity factor.
\begin{table}[htbp]
	\centering
	\caption{Estimates of the circuity factor $c$ and its corresponding $\beta_c$ (\cite{merchan2019empirical})}
	\label{tab:beta-merchan}
	\begin{tabular}{lccc}
		\toprule
		          & Area 1 & Area 2 & Area 3 \\
		\midrule
		$c$       & 2.76   & 2.34   & 1.82   \\
		$\beta_c$ & 2.48   & 2.10   & 1.64   \\
		\bottomrule
	\end{tabular}
\end{table}
It is important to note, however, that the assumptions in this study may limit the generality of
the findings. In particular, the use of uniformly distributed locations does not accurately reflect
the spatial distribution of delivery points in real urban environments, where locations tend to
cluster in residential, commercial, or industrial zones. Additionally, within small urban areas,
high-rise buildings and single-family homes may coexist in the same neighborhoods, further
challenging the assumption of uniformly distributed delivery points.
Furthermore, the circuity factor $c$ can
vary significantly within a single city, depending on local street patterns, infrastructure, and
topography. These variations suggest that a fixed circuity factor may oversimplify the complexity
of real-world delivery contexts, especially when applied to smaller sub-regions or neighborhoods.
\subsection{Lin-Kernighan Heuristic}
To be able to efficiently solve many TSPs, to find a good estimate for $\beta$, a fast and reliable
solution algorithm is needed. The Lin-Kernighan \cite{lin1973effective} heuristic provides outcome,
it is generally considered to be one of the most effective methods of generating (near) optimal
solutions for the TSP.
In this research a modified implementation of the heuristic is used \cite{helsgaun2000effective}.
The run times of both heuristics increase by approximately $n^{2.2}$, but the modified heuristic is
much more effective. It is able to find optimal solutions to large instances in reasonable times
\cite{helsgaun2000effective}.

\underbar{PARAGRAPH ABOUT HOW THE HEURISTIC WORKS}
\section{Experimental design}
\subsection{Data}
\subsection{Generation and solving of TSPs}
\section{Results}
\section{Discussion}
\section{Conclusion}
\bibliography{literature}
\bibliographystyle{rug-econometrics}
\section{Appendix}
\begin{longtable}{llcc}
\caption{Empirical estimates for $\beta$ in selected neighborhoods.} \label{tab:results}\\
\hline
Province & Neighborhood & $\beta$ & MAE \\
\hline
\endfirsthead
\hline
Province & Neighborhood & $\beta$ & MAE \\
\hline
\endhead
groningen & Hortusbuurt & 2.2049 & 0.0701 \\
groningen & Binnenstad & 2.0410 & 0.0734 \\
groningen & Oosterpoort & 2.0627 & 0.0604 \\
groningen & Rivierenbuurt & 1.7826 & 0.0497 \\
groningen & De Wijert & 1.7755 & 0.0638 \\
groningen & Oosterparkwijk & 1.7838 & 0.0516 \\
groningen & De Hoogte & 1.6922 & 0.0587 \\
groningen & Korrewegwijk & 2.0751 & 0.0537 \\
groningen & Schildersbuurt & 2.2203 & 0.0489 \\
groningen & Paddepoel & 1.6792 & 0.0481 \\
groningen & Oranjewijk & 1.9282 & 0.0545 \\
groningen & Tuinwijk & 2.9039 & 0.0469 \\
groningen & Selwerd & 1.5247 & 0.0478 \\
groningen & Vinkhuizen & 1.4803 & 0.0435 \\
groningen & Hoogkerk-zuid & 1.5621 & 0.0590 \\
groningen & Gravenburg & 1.2985 & 0.0853 \\
groningen & De Held & 1.9223 & 0.0409 \\
groningen & Reitdiep & 1.6478 & 0.0530 \\
groningen & Hoornse Meer & 1.5935 & 0.0447 \\
groningen & Corpus den Hoorn & 1.6088 & 0.0508 \\
groningen & Eemspoort & 1.7165 & 0.0448 \\
groningen & Euvelgunne & 1.9353 & 0.1067 \\
groningen & Driebond & 1.9220 & 0.0599 \\
groningen & Winschoterdiep & 1.9630 & 0.0513 \\
groningen & Eemskanaal & 1.7551 & 0.0418 \\
groningen & Helpman & 2.0721 & 0.0618 \\
groningen & Lewenborg & 1.9303 & 0.0521 \\
groningen & Beijum & 1.7897 & 0.0429 \\
groningen & Maarsveld & 1.6844 & 0.0393 \\
noord holland & Schrijverswijk & 1.7493 & 0.0424 \\
noord holland & Stad van de Zon & 1.4790 & 0.1543 \\
noord holland & Stadshart & 1.4533 & 0.0663 \\
noord holland & Jordaan & 1.8471 & 0.0591 \\
noord holland & Slotervaart & 1.7456 & 0.0437 \\
noord holland & IJburg & 1.3349 & 0.0506 \\
noord holland & Oostelijke Eilanden & 1.6866 & 0.0424 \\
noord holland & Oostelijk Havengebied & 1.7129 & 0.0464 \\
noord holland & Frederik Hendrikbuurt & 2.3020 & 0.0582 \\
noord holland & Van Lennepbuurt & 1.7810 & 0.0596 \\
noord holland & Da Costabuurt & 2.4887 & 0.0585 \\
noord holland & Kinkerbuurt & 1.9679 & 0.0585 \\
noord holland & Kersenboogerd & 1.6571 & 0.0526 \\
noord holland & Pax & 2.1825 & 0.0415 \\
noord holland & Graan voor Visch & 2.2300 & 0.0531 \\
noord holland & Vrijschot-Noord & 2.4427 & 0.0556 \\
noord holland & Toolenburg & 1.2971 & 0.0443 \\
noord holland & Floriande & 1.9124 & 0.0485 \\
noord holland & Overbos & 1.7727 & 0.0419 \\
noord holland & Bornholm & 1.8024 & 0.0399 \\
noord holland & Beukenhorst-Oost & 1.7106 & 0.0833 \\
noord holland & De Hoek & 2.5441 & 0.0554 \\
noord holland & West & 2.0281 & 0.0489 \\
noord holland & Zuid & 1.6462 & 0.0527 \\
noord holland & Oost & 1.8809 & 0.0560 \\
noord holland & Noord & 1.6555 & 0.0485 \\
noord holland & De President & 1.5015 & 0.1269 \\
noord holland & Graan voor Visch-Zuid & 1.7029 & 0.0549 \\
noord holland & Zuidwijk & 1.4000 & 0.0406 \\
noord holland & Buitenveldert-West & 1.1975 & 0.0671 \\
noord holland & Buitenveldert & 1.1503 & 0.0710 \\
noord holland & Apollobuurt & 1.7468 & 0.0617 \\
noord holland & Stadionbuurt & 1.5084 & 0.0638 \\
noord holland & Prinses Irenebuurt e.o. & 1.8819 & 0.0428 \\
noord holland & Hoofddorppleinbuurt & 1.6903 & 0.0704 \\
noord holland & Willemspark & 1.9422 & 0.0576 \\
noord holland & Schinkelbuurt & 1.6424 & 0.0814 \\
noord holland & Vondelparkbuurt & 1.2941 & 0.0521 \\
noord holland & Helmersbuurt & 1.8039 & 0.0540 \\
noord holland & Overtoomse Sluis & 1.9497 & 0.0560 \\
noord holland & Museumkwartier & 1.7178 & 0.0597 \\
noord holland & Rivierenbuurt & 1.6822 & 0.0507 \\
noord holland & IJselbuurt & 1.5712 & 0.0496 \\
noord holland & Scheldebuurt & 1.3686 & 0.0565 \\
noord holland & Rijnbuurt & 1.6555 & 0.0510 \\
noord holland & De Baarsjes & 1.7849 & 0.0578 \\
noord holland & Landlust & 1.7685 & 0.0518 \\
noord holland & Staatsliedenbuurt & 1.8338 & 0.0520 \\
noord holland & Spaarndammerbuurt & 2.2421 & 0.0540 \\
noord holland & De Pijp & 2.2745 & 0.0521 \\
noord holland & Grachtengordel & 1.7832 & 0.0561 \\
noord holland & Oud-Zuid & 1.4443 & 0.0577 \\
\hline
\end{longtable}

\end{document}
