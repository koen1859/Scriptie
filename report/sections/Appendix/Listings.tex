\subsection{Listings}
\begin{lstlisting}[caption={The query to extract roads inside a neighborhood.}, label={lst:sql-roads}]
-- First, get the neighborhood polygon, in the coordinate format we need.
WITH neighborhood AS (
    SELECT ST_Transform(way, 4326) AS geom
    FROM planet_osm_polygon
    WHERE place = 'quarter'
      AND name = '{neighborhood}'
),
-- Then, define the road geometries in a way that we can filter based 
-- on whether they are inside the neighborhood.
road_geometries AS (
    SELECT
	w.id AS road_id,
	w.nodes AS node_ids,
	w.tags->>'oneway' AS oneway,
	ST_MakeLine(ARRAY(
	    SELECT ST_SetSRID(
		ST_MakePoint(n.lon / 1e7, n.lat / 1e7), 4326
		)
	    FROM unnest(w.nodes) WITH ORDINALITY AS u(node_id, ordinality)
	    JOIN planet_osm_nodes n ON n.id = u.node_id
	    ORDER BY u.ordinality
	)) AS road_geom
    FROM planet_osm_ways w
    -- Also filter based on the road type.
    WHERE w.tags->>'highway' IN (
	'trunk', 'rest_area', 'service', 'secondary_link',
	'services', 'tertiary', 'primary', 'secondary',
	'tertiary_link', 'road', 'motorway', 'motorway_link', 
	'corridor', 'primary_link', 'residential', 'trunk_link', 
	'living_street', 'unclassified', 'proposed'
    )
),
-- Filter on whether the roads are at least partly in the neighborhood.
filtered_roads AS (
    SELECT rg.*
    FROM road_geometries rg, neighborhood nb
    WHERE
	ST_Intersects(rg.road_geom, ST_Buffer(nb.geom, 0.0001))
)
-- Select the attributes that are needed.
SELECT
    fr.road_id,
    array_agg(n.id ORDER BY u.ordinality) AS node_ids,
    array_agg(n.lat / 1e7) AS node_lats,
    array_agg(n.lon / 1e7) AS node_lons,
    fr.oneway
FROM filtered_roads fr
JOIN planet_osm_ways w ON fr.road_id = w.id
JOIN LATERAL unnest(w.nodes) 
    WITH ORDINALITY 
    AS u(node_id, ordinality) 
    ON true
JOIN planet_osm_nodes n ON n.id = u.node_id
GROUP BY fr.road_id, fr.oneway;
\end{lstlisting}
\begin{lstlisting}[language=Python, caption={The algorithm to extract the edges to connect the buildings to the road network}, label={lst:python-edges}]
tree = STRtree(road_segments)  # make a tree of the road network

# Counter to add new nodes to connect buildings to road network correctly
new_node_idx = 0
for building_id, building_coords in buildings.items():
    building_point = Point(building_coords)

    # find the nearest edge
    nearest_segment_idx = tree.nearest(building_point)  
    start_node, end_node, oneway = segment_info[nearest_segment_idx]

    # Project the building onto this nearest edge
    nearest_segment = road_segments[nearest_segment_idx]
    projected_point = nearest_segment.interpolate(
	nearest_segment.project(building_point)
    )

    # If the projected point is one of the segments end points, use this
    if projected_point.equals(Point(nodes[start_node])):
	connect_to = start_node
    elif projected_point.equals(Point(nodes[end_node])):
	connect_to = end_node
    # else, we need to create a virtual node
    else:
	virtual_node_id = f"virtual_{new_node_idx}"
	nodes[virtual_node_id] = (projected_point.x, projected_point.y)
	new_node_idx += 1

	# We split the road segment and add the virtual node
	edges.append((start_node, virtual_node_id))
	weights.append(
	  distance(nodes[start_node], nodes[virtual_node_id])
	  )
	edges.append((virtual_node_id, end_node))
	weights.append(
	  distance(nodes[virtual_node_id], nodes[end_node])
	  )

	if oneway != "yes":  # if not one-way, add reverse
	    edges.append((virtual_node_id, start_node))
	    weights.append(
	      distance(nodes[virtual_node_id], nodes[start_node])
	      )
	    edges.append((end_node, virtual_node_id))
	    weights.append(
	    distance(nodes[end_node], nodes[virtual_node_id])
	    )

	# And we need to connect the building to the newly created node
	connect_to = virtual_node_id

    # Finally, we make the connections
    edges.append((str(building_id), connect_to))
    weights.append(distance(building_coords, nodes[connect_to]))
    edges.append((connect_to, str(building_id)))
    weights.append(distance(nodes[connect_to], building_coords))
\end{lstlisting}
