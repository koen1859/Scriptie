\subsection{Data}
In order to model the complex nature of real road networks, data from OpenStreetMap \citep{openstreetmap} is used. 
OpenStreetMap is an open-source project that provides geographic data, including accurate and 
detailed information about roads, buildings and natural features around the world. The data is 
continuously maintained and updated by a large community of users, making it a valuable resource 
for this research.

This data can be downloaded from Geofabrik, and then exported to a \url{PostgreSQL} database using
\url{osm2pgsql} \citep{osm2pgsql}, in order to be able to efficiently use the data with \url{Python}. 
For this analysis, the database has three interesting tables: 
\url{planet_osm_polygon}, \url{planet_osm_nodes} and \url{planet_osm_ways}.

A large number of neighborhoods multiple towns and villages in the Netherlands have a polygon 
defined in the data. In OpenStreetMap a polygon is a closed shape formed by a set of geographic coordinates 
(\url{nodes}) that are connected by lines (\url{ways}). These objects can be used to define boundaries of
geographic areas, such as lakes, parks, nature reserves and parts of cities and villages. In this
research the polygons are used to filter the buildings and roads only in a certain area efficiently.
These polygons are stored in \url{planet_osm_polygon}.

In the database, the roads are defined as \url{ways}. These ways have three
attributes: \url{id}, \url{nodes} and \url{tags}. The attribute \url{nodes} contains an ordered list
of the nodes that this road contains of.
In the \url{tags}, a large amount of 
information about the way is stored, for instance whether it is a 
one-way road, or the type of road that it is, i.e. primary, or trunk.
The information about the roads that are needed for this analysis is the
road id, the ids of the nodes the road consists of, the coordinates
(Latitude, Longitude) of these nodes, and whether the road is a one-way
road.

The buildings are stored as \url{nodes}. In this table (\url{planet_osm_nodes}),
a large amount of other objects are stored as well. For this analysis,
the potential delivery locations need to be extracted. Some of these 
nodes have a postcode defined, which can be used to extract all buildings
that a potential delivery could take place. This way, for example a little shed in 
someones backyard is also filtered out, since this does not have its own 
postcode. For this research only the node id and the coordinates are
needed.
